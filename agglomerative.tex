\section{Main results}

Instead of the divisive approach, we consider here an agglomerative approach, 
%Algorithm~\ref{alg:aic}, that computes $`g_{\ell}$ and $\mcP_{\ell}$ iteratively from $\ell=N$
Algorithm~\ref{alg:aic}, that computes $`g_{\ell}$ and $\mcP_{\ell}$ by iteratively producing coarser partitions from finer ones, i.e., from $\mcP_N$ to $\mcP_1$. 
(Note that $N$ needs not be known a priori.)
%We will give an efficient implementation of the subroutine \Agglomerate in Algorithm~\ref{algo:fuse}
%that computes $`g_\ell$ and $\mcP_{\ell-1}$ from $\mcP_{\ell}$ for any $\ell$, hence resolving the
%inefficiency of the divisive approach while returning the clusters in order.  In particular, we will
%show that it suffices to compute
We will give an efficient implementation (thereby resolving the inefficiency of the divisive
approach) of the subroutine \Agglomerate in Algorithm~\ref{algo:fuse} that computes $`g_\ell$ and
$\mcP_{\ell-1}$ from $\mcP_{\ell}$ for any $\ell$ (thereby returning the clusters in the desired order).
In particular, we will show that it suffices to compute
\begin{subequations}
	\label{eq:IC*}
\begin{align}
	I^*(\RZ_V)&:=\max\{I(\RZ_{C})\mid C\subseteq V, \abs {C} > 1\} \kern1em \text{ and}\label{eq:I*}\\
	\begin{split}
	\pzC^*(\RZ_V)&:=\op{maximal}\{C\subseteq V \mid \abs{C} > 1, \\
	&\kern8em I(\RZ_C) = I^{*}(\RZ_{V})  \},
	\end{split}\label{eq:C*}
\end{align}
\end{subequations}
which are, respectively, the last critical value $`g_{N}$  \cite[Theorem~1]{chan16cluster} and the last set of clusters (i.e., the non-singleton elements of the second last
partition $\mcP_{N-1}$).
%%%%%%%%%%%%%%%%%%%%%%%%%%%%%%%
%These quantities turn out to have a fundamental connection
%with the normalized total correlation~\eqref{eq:JT}, which gives rise to the efficient
%implementation.
%%%%%%%%%%%%%%%%%%%%%%%%%%%%%%
Recall 
%the MMI 
%as in~
\eqref{eq:I} 
and note that 
if the partition $\mcP=\Set {\Set {i}\mid i\in V}$ is
optimal
%to \eqref{eq:I},
then the MMI is equal to
\begin{align}
	\label{eq:JT}
	\begin{split}
		J_{\opT}(\RZ_V)
		&:=I_{\Set{\Set{i}\mid i\in V}}(\RZ_V)\\
		%&=\frac1 {\abs{V}-1}`1[\sum\nolimits_{i \in V}H(\RZ_i)-H(\RZ_V)`2],
		&=\frac1 {\abs{V}-1}`1[\sum\nolimits_{i \in V}h(\Set{i})-h(V)`2],
	\end{split}
\end{align}
which is the normalized version (see~\cite{chan15mi} for more details) of
the well-known Watanabe's total correlation~\cite{watanabe60}.
%%%%%%%%%%%%%%%%%%%%%%%%%%%%%%
The quantities $I^{*}$ and $\pzC^{*}$ turn out to have a fundamental connection
with the normalized total correlation~\eqref{eq:JT}, which leads to the desired efficient
implementation.


\label{sec:results}
%\input{results}

%%%%%%%%%%%%%%%%%%%%%%%%%%%%%%%%%%%%%%%%%%%%%%%%%%%%%%
\begin{algorithm}
	\caption{Agglomerative info-clustering.}
	\label{alg:aic}
	\KwData{Statistics of $\RZ_V$ sufficient for calculating the entropy function $h(B)$ for
		$B\subseteq V:=\{1,\ldots,n\}$.}
	\KwResult{Two arrays \DataSty{L} and \DataSty{PSP} that contain the $`g_{\ell}$'s and $\mcP_{\ell}$'s
		in Proposition~\ref{prop:clusters}. More precisely,
		for $1\leq \ell \leq N$, the entries of the arrays
		\DataSty{L}
		and
		\DataSty{PSP}
		are 
		\DataSty{L}$[s] = `g_{\ell}$
		and
		\DataSty{PSP}$[s] = \mcP_{\ell}$
		for $s = |\mcP_{\ell}|$, 
		and are null otherwise.}
	\BlankLine
	\DataSty{L}, \DataSty{PSP}$\leftarrow$ empty arrays, each of length $n$;\;
	\DataSty{PSP}$[n]$ $\leftarrow \{\{i\}:i\in V\}$, $s\leftarrow n$;\;
	\While{$s>1$}{
		$(`g,\mcP')\leftarrow$ $\Agglomerate($\DataSty{PSP}$[s])$;\;
		\DataSty{L}$[s]\leftarrow `g$;
		$s\leftarrow |\mcP'|$;
		\DataSty{PSP}$[s]\leftarrow \mcP'$;\;
	}
	% 	\BlankLine
	%   	\myfn{\Agglomerate{$\mcP$}}{
	%			assume $\mcP=\mcP_{\ell}$ for $1\leq \ell\leq N$ in Proposition~\ref{prop:clusters};\;
	%			\KwRet $(`g,\mcP_{\ell}) \leftarrow (`g_{\ell},\mcP_{\ell-1})$;
	%   	}
	%	\textcolor{red}{---needs revision, particularly the function fuse.}
\end{algorithm}
%%%%%%%%%%%%%%%%%%%%%%%%%%%%%%%%%%%%%%%%%%%%%%%%%%%%%%

The main result is the implementation of \Agglomerate in Algorithm~\ref{algo:fuse} that computes the
PSP, and so the info-clustering solution, iteratively from finer partitions to coarser
ones. This is done by computing the PS using a subroutine \MinNormBase that computes the minimum
norm base for \eqref{eq:MNB}. An explicit implementation of \MinNormBase can be found in
\cite{fujishige11} using Wolfe's minimum norm point algorithm.
%The current abstraction also allows further approximations or simplifications for special source models as in \cite{chan16cluster}. 

\begin{algorithm}
	\caption{Implementation of \texttt{agglomerate}.}% in Algorithm~\ref{alg:aic}.}
	\label{algo:fuse}
	%	\KwData{Statistics of $\RZ_V$ sufficient for calculating \FirstClusters{$B$} for all $B\subseteq V:\abs{B}>1$.}
	%	\KwResult{$\mcS$ is a list of $(I(\RZ_C),C)$ for $C\in \pzC(\RZ_V)$, which gives
	%		$\Upgamma(\RZ_V)=\Set{`g' \mid (`g',B)\in \mcS}$ and
	%		$\pzC_{`g}(\RZ_V)=\op{maximal}\Set{B\mid (`g',B)\in \mcS,`g'>`g}$.}
	%	\BlankLine
	\BlankLine
	\KwData{$\mcP$ is equal to $\mcP_{\ell}$ for some $1\leq \ell\leq N$ in \eqref{eq:PSP}.} %Proposition~\ref{prop:clusters}.}
	\KwResult{$(`g,\mcP')$ is equal to $(`g_\ell,\mcP_{\ell-1})$.}
	%\myfn{\Agglomerate{$\mcP$}}{
		Enumerate $\mcP$ as $\Set{C_1,\dots,C_k}$ for some $k>1$ and disjoint $C_i$'s;\;
		\DataSty{x}$\leftarrow$ empty array of size $k$;\;
		\For{$j\leftarrow 1$ \emph{\KwTo} $k$ }{
			%\DataSty{x}[$j$]=\MinNormBase($ B\mapsto h`1(\bigcup_{i\in B\cup\Set{j}}C_i`2)-\sum_{i\in B\cup\Set{j}}h(C_i),\Set{j+1,\ldots,k}$);\; \label{ln:MNB:1}
			Define $g_j$ as the function $ B\subseteq \{j+1,\dots,k\}  \mapsto h`1(\bigcup_{i\in B\cup\Set{j}}C_i`2)-\sum_{i\in B\cup\Set{j}}h(C_i)$;\; 
			\DataSty{x}[$j$]$\leftarrow$\MinNormBase($g_j,\Set{j+1,\ldots,k}$);\; \label{ln:MNB:1}
		}
		%\KwRet $(\upgamma_1(\RZ_B),\pzC_{\upgamma_1}(\RZ_B))$;
		$\displaystyle`g\leftarrow -\min_{i,j:1\leq j<i\leq k}\DataSty{x}[j][i]$,\kern1em$\mcP'\leftarrow`0$;\;
		\For{$j \leftarrow 1$ \emph{\KwTo} $k$ }{
			\If{$C_j\nsubseteq \bigcup\mcP'$}{
				%add $\Set{C_j}\cup\Set{C_i\mid i\kern-.25em\in\kern-.25em\Set{j\kern-.25em +\kern-.25em 1,\dots,k}, \DataSty{x}[j][i]\kern-.25em\leq\kern-.25em -`g}$ to $\mcP'$;\;
				add $C_j \cup \bigcup\Set{C_i\mid i\kern-.25em\in\kern-.25em\Set{j\kern-.25em
				+\kern-.25em 1,\dots,k}, \DataSty{x}[j][i]\kern-.25em\leq\kern-.25em -`g}$ to $\mcP'$;\;
				%\tcc{see below \eqref{} for the notation ``$\bigcup$''}
			}
%			\If{$C_j\not\in \bigcup_{C\in \mcP'}C'$}{
%				add $\Set{C_j}\cup\Set{C_i\mid i\kern-.25em\in\kern-.25em\Set{j\kern-.25em +\kern-.25em 1,\dots,k}, \DataSty{x}[j][i]\kern-.25em\leq\kern-.25em -`g}$ to $\mcP'$;\;
%			}
			
		}
		
	%}
	\myfn{\MinNormBase($f,U$)}{
		%\tcc{ assume normalized submodular function $f:2^U\mapsto`R$ }
		\KwRet an array \DataSty{x} (indexed by $U$) that solves~\eqref{eq:MNB}.
	}
\end{algorithm}

%%%%%%%%%%%%%%%%%%%%%%%%%%%%%
%
%The complexity of the algorithm is mainly due to the computation of the minimum norm base in
%line~\ref{ln:MNB:1}.
%%The number of such computation is at most $n$.
%This computation is repeated at most $n$ times.
%With $\op{MNB}(n)$ being
%the complexity of the minimum norm base algorithm for ground set of size $n$, \Agglomerate runs in
%time $O(n\op{MNB}(n))$. Since the agglomerative info-clustering algorithm in
%Algorithm~\ref{alg:aic} invokes function \Agglomerate $N-1\leq n-1$ times, it runs in time
%$O(n^2 \op{MNB}(n))$, which is equivalent to that of~\cite[Algorithm~3]{chan16cluster},
%assuming that the submodular function minimization therein is implemented by the minimum norm base
%algorithm, i.e., with $\op{SFM}=\op{MNB}$. The divisive info-clustering algorithm
%in~\cite[Algorithm~2]{chan16cluster} makes $N-1$ calls to a subroutine that calculates the
%fundamental partition. However, computing the fundamental partition appears to take time
%$O(n^2 \op{MNB}(n))$, which would lead to an overall complexity of $O(n^3
%\op{MNB}(n))$ for the divisible clustering. Hence, the agglomerative info-clustering is
%more efficient.
%
%%%%%%%%%%%%%%%%%%%%%%%%%%%%%

To explain Algorithm~\ref{algo:fuse}, we need the following two fundamental results that
\begin{inparaenum}
\item 
relate the last critical value of a set of RVs to the normalized total correlation and,
%as a result,
consequently,
to the minimum norm base; and
\item 
relate any critical value of a set of RVs to the last critical value of a set of agglomerated RVs.
\end{inparaenum}
Each theorem will be illustrated with an example. %that is followed by a proof sketch.
The statements of the results may be
skimmed first and read more carefully as we go through the examples.
%The detailed proofs can be found in~\cite{chan17aic}.

\begin{Theorem}
	\label{thm:JT}
	With $V:= \{1,\ldots, n\}$ and $h$~\eqref{eq:h} denoting the entropy function of $\RZ_V$, let,
	for any $j\in V$,
	\begin{align*}
		U_j &:=\Set {i\in V\mid i>j} \kern0em \text{ and } \\ %\text{ assuming $V$ is ordered.}\\
	g_j(B) &:= h(B\cup \Set {j})-\sum\nolimits_{i\in B\cup \Set {j}} h(\Set {i}) \kern1em \text {for }B\subseteq U_j,
	\end{align*}
	which is a normalized submodular function. Then,
	\begin{subequations}
		\begin{align}
		I^*(\RZ_V)&=\max_{C\subseteq V:\abs{C}> 1} J_{T}(\RZ_C) \label{eq:maxJT}\\
		&= -\min_{j\in V} \min_{i\in U_j} x^{(j)}_i \label{eq:maxJT:x}\\ \nextParentEquation
		\pzC^*(\RZ_V)&=\op{maximal}
		\argmax_{C\subseteq V:\abs{C}> 1} J_{T}(\RZ_C)
		\label{eq:argmaxJT}\\
		&=\op{maximal} 
		`1\{`1.\Set{j^*}\cup 
		\argmin_{i\in U_{j^*}} x^{(j^*)}_i `2|`2.  \notag\\
		&\kern6.5em 
		`1.j^*\in \argmin_{j\in V} `1( \min_{i\in U_{j}} x^{(j)}_i `2)
		`2\},\label{eq:argmaxJT:x}
		\end{align}
	\end{subequations}
	where $x^{(j)}_{U_{j}}$ is the minimum norm base for $g_j$~\eqref{eq:MNB}.
\end{Theorem}
\begin{Proof}
	See Appendix~\ref{sec:proof}.
\end{Proof}

\begin{example}
	\label{eg:IC*}
	%As an illustration of Theorem~\ref{thm:JT}, 
	Consider our running example~\eqref{eq:eg-motivate}
	with $V = \Set{1,\ldots,6}$. By brute-force search, the maximum normalized total correlation in
	\eqref{eq:maxJT} is $J_{\opT}(\RZ_{\Set {1,2}})=2$ achieved uniquely by $C=\Set {1,2}$. Hence,
	\eqref{eq:maxJT} and \eqref{eq:argmaxJT} give, respectively, $I^*(\RZ_V)=2$ and $\pzC^*(\RZ_V)=\Set {\Set
		{1,2}}$, which correspond to the last critical value and last set of clusters
		in \eqref{eq:eg-clusters}, as desired by the definitions of $I^*$ and $\pzC^*$ in
		\eqref{eq:IC*}. 
	
	To illustrate \eqref{eq:maxJT:x} and \eqref{eq:argmaxJT:x},  
	the minimum norm base $x_{U_{j}}^{(j)}$ of $g_{j}$, for different values of $j$, is calculated as:
	\begin{align*}
	x_{U_j}^{(j)} =
	\scalebox{0.8}{
		\begin{blockarray}{lcc ccl c}
		\phantom{n}1 &	2 & 3 & 4 & 5 & 6 \\
		\begin{block}{\{lcc ccc c}
		(&	\circled{-2}, & -1, & -0.5, & -0.5, & 0),   & j=1 \\
		(&	   & -1,  & -0.5, & -0.5, & 0),   & j=2 \\
		(&	   &      & -0.5, & -0.5, & 0),   & j=3 \\
		(&	   &      &      & -1,   & 0),   & j=4 \\
		(&	   &      &      &      & 0),   & j=5. \\
		\end{block}
		\end{blockarray}}
	\end{align*}
%	For instance, when $j=4$, we have $U_j=\Set {5,6}$, $g_j(`0)=h(\Set {j})-h(\Set {j})=0$ and
%	$g_j(\Set {5})=h(\Set {4,5})-h(\Set {4})-h(\Set {5})=-I(\RZ_4\wedge \RZ_5)=-1$. Similarly,
%	$g_j(\Set {6})=-I(\RZ_4\wedge \RZ_6)=0$. $g_j(\Set {5,6})=h(\Set {4,5,6})-h(\Set {4})-h(\Set
%	{5})-h(\Set {6})=-1$. From \eqref{eq:PB}, $\rmP(g_j)=\Set {(x_5,x_6)\in `R^2\mid x_5\leq -1,
%	x_6\leq 0}$ and so $\rmB(g_j)=\Set {(-1,0)}$ contains a unique base, which is  therefore the
%	minimum norm base trivially. 
%%
%	For instance, when $j=4$, we have $U_j=\Set {5,6}$ and $g_j(`0)=0$,
%	$g_j(\Set {5})=h(\Set {4,5})-h(\Set {4})-h(\Set {5})=-I(\RZ_4\wedge \RZ_5)=-1$,
%	$g_j(\Set {6})=-I(\RZ_4\wedge \RZ_6)=0$, and $g_j(\Set {5,6})=h(\Set {4,5,6})-h(\Set {4})-h(\Set
%	{5})-h(\Set {6})=-1$. From \eqref{eq:PB}, $\rmP(g_j)=\Set {(x_5,x_6)\in `R^2\mid x_5\leq -1,
%	x_6\leq 0}$, and so $\rmB(g_j)=\Set {(-1,0)}$ contains a unique base, which is  therefore the
%	minimum norm base trivially. 
%	
	Now, the minimum in \eqref{eq:maxJT:x} is equal to $-2$ (circled above), which is achieved uniquely by $j=1$ and $i=2$. This gives $I^*(\RZ_V)=-(-2)=2$ as desired. In \eqref{eq:argmaxJT:x}, we have $j^*\in \Set {1}$ and $\argmin_{i\in U_{j^*}} x_i^{(j^*)}=\Set {2}$. Hence, $\pzC^*(\RZ_V)=\op{maximal}\Set {\Set {1}\cup \Set {2}}=\Set {\Set {1,2}}$ as desired.	
	%
\end{example}


We remark that \eqref{eq:maxJT} and~\eqref{eq:argmaxJT} eliminate the need for minimization over
partitions in calculating the MMI in~\eqref{eq:I*} and~\eqref{eq:C*}. They serve as an intermediate
step that leads to \eqref{eq:maxJT:x} and \eqref{eq:argmaxJT:x}, thereby relating the last critical value \eqref{eq:I*} and the last set of clusters 
\eqref{eq:C*} to the minimum norm base. 

%\begin{Proof}[Sketch for Theorem~\ref{thm:JT}]
%	
%\eqref{eq:maxJT} and \eqref{eq:argmaxJT} can be proved by showing that the optimal partition for any
%maximizer $C$ to \eqref{eq:I*}  is the partition $\Set {\Set {i}\mid i\in C}$ into singletons, in
%which case $I(\RZ_C)=J_{\opT}(\RZ_C)$. 
%%\eqref{eq:maxJT} and \eqref{eq:argmaxJT} can be proved by showing that,
%%for any maximizer $C$ to \eqref{eq:I*}, 
%%the optimal partition achieving $I(\RZ_{C})$
%%is the partition $\Set {\Set {i}\mid i\in C}$ into singletons, and so $I(\RZ_C)=J_{\opT}(\RZ_C)$. 
%
%To obtain the minimum norm base formulation in \eqref{eq:maxJT:x} and \eqref{eq:argmaxJT:x}, the
%important step is to rewrite \eqref{eq:maxJT} as
%%\begin{align*}
%%0&=\min_{C\subseteq V:\abs{C}>1}`g^*-J_{\op{T}}(\RZ_C)\\
%%&=\min_{C\subseteq V:\abs{C}>1}(\abs{C}-1)`g^*+h(C)-\sum_{i\in C}h(\Set{i})  \\
%%&=\min_{j\in V}\min_{B\subseteq U_j:\abs{B}\geq1} g_j(B)+`g^*\abs{B}.
%%\end{align*}
%\begin{align*}
%	0&=\min_{C\subseteq V:\abs{C}>1} \big[I^*(\RZ_V)-J_{\op{T}}(\RZ_C) \big]\\
%	&=\min_{C\subseteq V:\abs{C}>1} \big[(\abs{C}-1) I^*(\RZ_V)+h(C)-\sum_{i\in C}h(\Set{i}) \big]  \\
%	&=\min_{j\in V}\min_{B\subseteq U_j:\abs{B}\geq1} g_j(B)+I^*(\RZ_V)\abs{B}.
%\end{align*}
%In particular, the inner minimization can be cast as the minimization in \eqref{eq:S_`l} for the PS, and the solution via the minimum norm base in \eqref{eq:MNB:S_`l} gives \eqref{eq:maxJT:x} and \eqref{eq:argmaxJT:x}.
%\end{Proof}



So far we succeeded in posing the computation of the last critical value $I^{*}$ and
last set of clusters $\pzC^{*}$ as a minimum norm base problem.
However, our objective is to compute the entire clustering solution.
To complete the implementation of \Agglomerate, we need the following theorem which reduces
the computation of the clusters at any threshold to the computation of $I^*$ and $\pzC^*$ for some
agglomerated RVs.
%, we have the complete implementation of \Agglomerate as shown in Algorithm~\ref{algo:fuse} using a minimum norm base algorithm.

%%%%%%%%%%%%%%%%%%%%%%%%%%%%%%%%%%%%%%%%%%%%%%%%%%%%%%%%%%%%%%%%%%%%%%%%%%
\begin{Theorem}
	\label{thm:`g:P:fused}
 Consider  $\mcP_{\ell}$ and $`g_\ell$ as defined in Proposition~\ref{prop:clusters} and, for $1\leq
 \ell\leq N$, write
 \begin{align}
 	\mcP_{\ell}=\Set{C_j\mid j\in U} \kern1em \text{and}\kern1em
 	\RZ'_j:=\RZ_{C_j} \label {eq:Z'}
 \end{align}
 for some index set $U$ and disjoint subsets $C_j$ for $j\in U$. Then, for $1\leq \ell \leq N$, we
 have
 \begin{subequations}[eq:A:`g]
 	\begin{align}
 	`g_\ell
	&=\max_{\mcF\subseteq \mcP_{\ell}:\abs{\mcF}>1} I(\RZ_{\bigcup \mcF}) \label{eq:A:`g_l}\\
 	&=I^*(\RZ'_U), %:=\max`1\{I(\RZ'_B)\mid B\subseteq U,\abs{B}>1`2\}
 	\label{eq:A:`g_l:fused}\\
 	\nextParentEquation[eq:A:P]
 	\kern-.7em\mcP_{\ell-1}`/\mcP_{\ell}
 	&=\op{maximal}\Set*{\bigcup\mcF`1| \mcF\in\kern-.5em \argmax_{\mcF\subseteq \mcP_{\ell}:\abs{\mcF}>1} \kern-.5em I(\RZ_{\bigcup \mcF}) \kern-.3em `2.}\kern-.5em\label{eq:A:P_i-1}\\
 	&=`1\{\bigcup\nolimits_{j\in B}C_j \mid B\in \pzC^*(\RZ'_U)`2\}, \label{eq:A:P_i-1:fused}
 	\end{align}
 \end{subequations}
where $\bigcup \mcF:=\bigcup_{B\in \mcF} B$ for convenience.
%More precisely, \eqref{eq:A:`g_l} and~\eqref{eq:A:P_i-1} follow more generally from the property~\eqref{eq:imunion}, while \eqref{eq:A:`g_l:fused} and \eqref{eq:A:P_i-1:fused} follow from 
%the definition~\eqref{eq:mmi} of the MMI.
\end{Theorem}
\begin{Proof}
	See Appendix~\ref{sec:proof}.
\end{Proof}
%%\newcommand{\cb}{\bigcup_{i\in B}C_i}
%%\newcommand{\cbb}{\bigcup\limits_{i\in B'}C_i}
%\newcommand{\cb}{C_B}
%\newcommand{\cbb}{C_{B'}}
%\begin{Theorem}
%	\label{thm:`g:P:fused}
% Consider $\mcP_{\ell}$ and $`g_\ell$ defined as in Proposition~\ref{prop:clusters} for $1\leq \ell\leq N$ and write
% \begin{align}
% 	\mcP_{\ell}=\Set{C_j\mid j\in U} \kern1em \text{and}\kern1em
% 	\RZ'_j:=\RZ_{C_j} \label {eq:Z'}
% \end{align}
% for some index set $U$ and disjoint subset $C_j$ for $j\in U$. Then, for $1\leq \ell \leq N$ we
% have
% \begin{subequations}[eq:A:`g]
% 	\begin{align}
%		`g_\ell&=\max_{B\subseteq U,\abs{B}>1} I(\RZ_{\cb}) \label{eq:A:`g_l}\\
% 	&=I^*(\RZ'_U)%:=\max`1\{I(\RZ'_B)\mid B\subseteq U,\abs{B}>1`2\}
% 	\label{eq:A:`g_l:fused}\\
% 	\nextParentEquation[eq:A:P]
% 	\kern-.7em\mcP_{\ell-1}`/\mcP_{\ell}
%	&=\op{maximal}\Set*{C_{B}`1| B\in\kern-.5em \argmax_{B'\subseteq U:\abs{B'}>1} \kern-.5em I(\RZ_{\cbb}) \kern-.3em `2.}\kern-.5em\label{eq:A:P_i-1}\\
%	&=`1\{\cb \mid B\in \pzC^*(\RZ'_U)`2\}, \label{eq:A:P_i-1:fused}
% 	\end{align}
% \end{subequations}
%where $C_B:=\bigcup_{i\in B} C_i$ for convenience.
%%More precisely, \eqref{eq:A:`g_l} and~\eqref{eq:A:P_i-1} follow more generally from the property~\eqref{eq:imunion}, while \eqref{eq:A:`g_l:fused} and \eqref{eq:A:P_i-1:fused} follow from 
%%the definition~\eqref{eq:mmi} of the MMI.
%\end{Theorem}

%\begin{Proof}
%	See Appendix~\ref{sec:proof}.
%\end{Proof}
%%%%%%%%%%%%%%%%%%%%%%%%%%%%%%%%%%%%%%%%%%%%%%%%%%%%%%%%%%%%%%%%%%%%%%%%%%



%It follows from~\eqref{eq:A:`g_l:fused} and~\eqref{eq:A:P_i-1:fused} that it suffices to compute $I^*$~\eqref{eq:I*} and $\pzC^*$~\eqref{eq:C*}. This can be done using a minimum norm base algorithm as follows. Define
%\begin{align}
%\label{eq:JT}
%\begin{split}
%J_{\opT}(\RZ_V)&:=I_{\Set{\Set{i}\mid i\in V}}(\RZ_V)\\
%&=\frac1 {\abs{V}-1}`1[\sum_{i \in V}H(\RZ_i)-H(\RZ_V)`2],
%\end{split}
%\end{align}
%which is called the normalized total correlation~\cite{chan15mi} as it is the same as Watanabe's total correlation~\cite{watanabe60} except for the additional normalization factor of $\frac 1{\abs {V}-1}$.

%\begin{Remark}

%For efficient implementation, we can further impose $i$ in~\eqref{eq:I^*:f} to be the smallest element in $B\cup\Set{i}$ without loss of optimality, and so the minimization can be done over a smaller set of $B\subseteq \Set{j\in V\mid j> i}$. (See line~\ref{ln:MNB:1} of Algorithm~\ref{algo:fuse}.)
%\end{Remark}

%%%%%%%%%%%%%%%%%%%%%%%%%%%%%
\begin{example}
	%As an illustration of Theorem~\ref{thm:`g:P:fused} (and the agglomerative algorithm), 
	We now illustrate Theorems~\ref{thm:JT} and \ref{thm:`g:P:fused} using our running
	example~\eqref{eq:eg-motivate}.

	In the first iteration, $l=N$, consider \eqref{eq:Z'} with $\mcP_{\ell}$ being the partition into
	singletons, namely, let $\RZ'_i=\RZ_i$ for $i=1, \dots, 6$ (i.e., $C_i=\Set {i}$).
	By \eqref{eq:maxJT:x} and \eqref{eq:argmaxJT:x},
	we have $I^{*}(\RZ_{\Set{1,\ldots,6}}) = 2$ and $\pzC^*(\RZ_{\Set{1,\dots,6}})=\Set {\Set {1,2}}$. (See Example~\ref{eg:IC*}.)
	By \eqref{eq:A:`g_l:fused} and \eqref{eq:A:P_i-1:fused},
	we have $`g_{N}=2$ and $\mcP_{N-1}`/\mcP_{N}=\Set{\Set{1,2}}$ as desired. 
	In other words, when called with the
	singletons partition
	as the input partition, the function \Agglomerate returns the threshold value $`g_{N}$ and partition
	$\mcP_{N-1}$ correctly.
	
	In the next iteration, $l=N-1$, consider \eqref{eq:Z'} with $\mcP_{\ell}$ as computed above,
	namely, let $\RZ'_1 = \RZ_{\Set{1,2}}$ and $\RZ'_{i} = \RZ_{i+1}$ for
	$i=2,\ldots,5$ (i.e., $C_1=\Set {1,2}$ and $C_i=\Set {i+1}$).
	The minimum norm base $x_{U_{j}}^{(j)}$ of $g_{j}$ (defined using the entropy function of $\RZ'_{\Set{1,\dots,5}}$) is
	\begin{align*}
	x_{U_j}^{(j)} =\scalebox{0.8}{
	\begin{blockarray}{lcc cl c}
		\phantom{n}1 & 2 & 3 & 4 & 5 \\
					\begin{block}{\{lcc cc c}
							 (& \circled{-1}, & -0.5, & -0.5, & 0),   & j=1 \\
							 (&     & -0.5, & -0.5, & 0),   & j=2 \\
							 (&     &       & \circled{-1},   & 0),   & j=3 \\
							 (&     &       &       & 0),   & j=4 \\
					\end{block}
	\end{blockarray}}
	\end{align*}
	By \eqref{eq:maxJT:x} and \eqref{eq:argmaxJT:x},
	we have $I^{*}(\RZ'_{\Set{1,\ldots,5}}) = 1$ and 
	$\pzC^{*}(\RZ'_{\Set{1,\ldots,5}}) = \Set{\Set{1,2},\Set{3,4}}$, which in terms of
	$\RZ_{V}$, results in the clusters $\Set{\Set{1,2,3},\Set{4,5}}$.
	By \eqref{eq:A:`g_l:fused} and \eqref{eq:A:P_i-1:fused},
	we have $`g_{N-1}=1$ and $\mcP_{N-2}`/\mcP_{N-1}=\Set{\Set{1,2,3},\Set{4,5}}$ as desired. 
	In other words, when called with the input partition $\Set{\Set{1,2},\Set{3},\Set{4},\Set{5},
	\Set{6}}$, the function \Agglomerate returns the threshold value $`g_{N-1}$ and partition
	$\mcP_{N-2}$ correctly.

	In the next iteration, $l=N-2$, consider \eqref{eq:Z'} with $\mcP_{\ell}$ as computed above,
	namely, let $\RZ'_1 = \RZ_{\Set{1,2,3}}$ ($C_1=\Set {1,2,3}$), $\RZ'_{2} = \RZ_{\Set{4,5}}$ ($C_2=\Set {4,5}$), and $\RZ'_{3} = \RZ_{6}$ ($C_3=\Set {6}$).
	The minimum norm base $x_{U_{j}}^{(j)}$ of $g_{j}$ (defined using the entropy function of
	$\RZ'_{\Set{1,2,3}}$) is given as
	\begin{align*}
		x_{U_j}^{(j)} =\scalebox{0.8}{
		\begin{blockarray}{rc c c}
			\phantom{n}1 & 2   & 3 \\
						\begin{block}{\{l cc c}
							(& \circled{0},   & \circled{0}),   & j=1 \\
							(&     &               \circled{0}),   & j=2 \\
						\end{block}
		\end{blockarray}}
		\end{align*}
	By \eqref{eq:maxJT:x} and \eqref{eq:argmaxJT:x}, we have
	$I^{*}(\RZ'_{\Set{1,2,3}}) = 0$ and $\pzC^{*}(\RZ'_{\Set{1,2,3}}) = \Set{1,2,3}$, which in terms of
	$\RZ_{V}$, results in the trivial cluster $\Set{1,\dots,6}$.
	By \eqref{eq:A:`g_l:fused} and \eqref{eq:A:P_i-1:fused},
	$`g_{N-2}=0$ and $\mcP_{N-3} \backslash \mcP_{N-2} = \Set{1,\ldots,6}$ as desired.
	%When called with the input partition 
	%$\Set{\Set{1,2,3}, \Set{4,5}, \Set{6}}$, the function \Agglomerate will return the threshold value
	%given by \eqref{eq:A:`g} and \eqref{eq:A:P} respectively.
	At this point the agglomerative algorithm
	terminates.
\end{example}


The complexity of \Agglomerate (Algorithm~\ref{algo:fuse}) is mainly due to the computation of the
minimum norm base in line~\ref{ln:MNB:1}.  This computation is repeated at most $n$ times.  With
$\op{MNB}(n)$ denoting the complexity of the minimum norm base algorithm for a ground set of size $n$,
\Agglomerate runs in time $O(n\op{MNB}(n))$. Since the agglomerative info-clustering algorithm in
Algorithm~\ref{alg:aic} invokes function \Agglomerate $N-1\leq n-1$ times, it runs in time $O(n^2
\op{MNB}(n))$, which is equivalent to that of~\cite[Algorithm~3]{chan16cluster}, assuming that the
submodular function minimization therein is implemented by the minimum norm base algorithm, i.e.,
with $\op{SFM}=\op{MNB}$. In contrast, the divisive info-clustering algorithm
in~\cite[Algorithm~2]{chan16cluster} makes $N-1$ calls to a subroutine that calculates the
fundamental partition. However, computing the fundamental partition appears to take time $O(n^2
\op{MNB}(n))$, which leads to an overall complexity of $O(n^3 \op{MNB}(n))$ for the divisive
clustering.

%\begin{Proof}[Sketch for Theorem~\ref{thm:`g:P:fused}]
%	To prove \eqref{eq:A:`g_l}, we first argue that, for any feasible $\mcF$ to the r.h.s.\ of~\eqref{eq:A:`g_l}, 
%	\begin{align*}
%	I(\RZ_{\bigcup\mcF})\utag{i}\leq `g_\ell.
%	\end{align*}
%	Suppose to the contrary that $I(\RZ_{\bigcup\mcF})>`g_{\ell}$. Then, there exists
%	$C\supseteq\bigcup \mcF$ such that $C\in \pzC_{`g_\ell}(\RZ_V)$ by 
%	definition~\eqref{eq:clusters} of clusters.
%	Hence, 
%	$C \in \mcP_{\ell}$ since
%	$\pzC_{`g_\ell}(\RZ_V)\subseteq \mcP_{\ell}$
%	by Proposition~\ref{prop:clusters}, which contradicts the fact that $\mcF\subseteq \mcP_{\ell}$
%	with $\abs {\mcF}>1$. 
%%	However, $\pzC_{`g_\ell}(\RZ_V)\subseteq \mcP_{\ell}$
%%	by Proposition~\ref{prop:clusters}, which contradicts the fact that $\mcF\subseteq \mcP_{\ell}$
%%	with $\abs {\mcF}>1$. 
%	
%		Next, we show that \uref{i} can be achieved with equality for some feasible solution $\mcF$.
%		Consider any $C\in \mcP_{\ell-1}`/\mcP_{\ell}$. (Such a $C$ exists since $\mcP_{\ell}$ is
%		strictly finer than $\mcP_{\ell-1}$.) Then, we have 
%	\begin{align*}
%	C \utag{ii}= \bigcup \mcF\text{ for some }\mcF \subseteq \mcP_{\ell}:|\mcF| > 1,
%	\end{align*}
%	i.e., for some feasible solution $\mcF$.
%	%By Proposition~\ref{prop:clusters}, we have 
%	%$C\in \pzC_{`g}(\RZ_V)$ for all
%	%$`g\in [`g_{\ell-1},`g_\ell)$,
%	%and so $I(\RZ_C)\geq `g_\ell$. %, i.e., larger than all values in the interval.
%	%The reverse inequality also holds by \uref{i} and \uref{ii}. Hence,
%	%%\begin{align*}
%	%$I(\RZ_C)=`g_\ell$,
%	%%\end{align*}
%	%which implies \eqref{eq:A:`g_l} as desired. 
%	Proposition~\ref{prop:clusters} states that 
%	$C\in \pzC_{`g}(\RZ_V)$ for all
%	$`g\in [`g_{\ell-1},`g_\ell)$ since $C\in \mcP_{l-1}$,
%		and so $I(\RZ_C)\geq `g_\ell$ by \eqref{eq:clusters}.
%	The reverse inequality also holds by \uref{i} and \uref{ii}. Hence,
%	%\begin{align*}
%	$I(\RZ_C)=`g_\ell$,
%	%\end{align*}
%	which implies \eqref{eq:A:`g_l} as desired. 
%
%	\eqref{eq:A:P_i-1} can be proved by showing that the above construction gives all the optimal solutions to the r.h.s.\ of
%	\eqref{eq:A:`g_l}.
%	
%%	Now, we argue that the above construction gives all the optimal solutions to the r.h.s.\ of
%%	\eqref{eq:A:`g_l}, hence establishing \eqref{eq:A:P_i-1}. For any $C\in
%%	\mcP_{\ell-1}`/\mcP_{\ell}$, \uref{ii} and \uref{iii} imply that ``$\subseteq$'' holds
%%	for~\eqref{eq:A:P_i-1}, because the fact that $C\in \pzC_{`g_{\ell-1}}(\RZ_V)$ (by
%%	Proposition~\ref{prop:clusters}) means that it is maximal by the definition~\eqref{eq:clusters}
%%	of  clusters. 
%%	
%%	To argue the reverse inclusion ``$\supseteq$'', consider any $\mcF$ belonging to the r.h.s.\
%%	of~\eqref{eq:A:P_i-1}. By \eqref{eq:A:`g_l}, 
%%	%\begin{align*}
%%	$I(\RZ_{\bigcup \mcF})=`g_\ell>`g_{\ell-1}$
%%	%\end{align*}
%%	and so $\bigcup \mcF\in \pzC_{`g_{\ell-1}}$ by the definition~\eqref{eq:clusters} of clusters and
%%	the maximality of $\bigcup \mcF$. This completes the poof of~\eqref{eq:A:P_i-1}.
%%	%To prove \eqref{eq:A:P_i-1}, 
%	
%	\eqref{eq:A:`g_l:fused} and \eqref{eq:A:P_i-1:fused} can be proved using the definitions of $I^*$
%	and $\pzC^*$ in~\eqref{eq:IC*} and Proposition~\ref{prop:fund:cluster}.
%\end{Proof}

%\begin{Proof}[Sketch for Theorem~\ref{thm:`g:P:fused}]
%	To prove \eqref{eq:A:`g_l}, we first argue that, for any feasible $\mcF$ to the r.h.s.\ of~\eqref{eq:A:`g_l}, 
%	\begin{align*}
%	I(\RZ_{\bigcup\mcF})\utag{i}\leq `g_\ell.
%	\end{align*}
%	Suppose to the contrary that $I(\RZ_{\bigcup\mcF})>`g_{\ell}$. Then, there exists
%	$C\supseteq\bigcup \mcF$ such that $C\in \pzC_{`g_\ell}(\RZ_V)$ by the
%	definition~\eqref{eq:clusters} of clusters. However, $\pzC_{`g_\ell}(\RZ_V)\subseteq \mcP_{\ell}$
%	by Proposition~\ref{prop:clusters}, which contradicts the fact that $\mcF\subseteq \mcP_{\ell}$
%	with $\abs {\mcF}>1$. 
%	
%		Next, we show that \uref{i} can be achieved with equality for some feasible solution $\mcF$.
%		Consider any $C\in \mcP_{\ell-1}`/\mcP_{\ell}$. (Such a $C$ exists since $\mcP_{\ell}$ is
%		strictly finer than $\mcP_{\ell-1}$.) Then, we have 
%	\begin{align*}
%	C \utag{ii}= \bigcup \mcF\text{ for some }\mcF \subseteq \mcP_{\ell}:|\mcF| > 1,
%	\end{align*}
%	i.e., for some feasible solution $\mcF$.
%	
%	By Proposition~\ref{prop:clusters}, we have 
%	$C\in \pzC_{`g}(\RZ_V)$ for all
%	$`g\in [`g_{\ell-1},`g_\ell)$,
%	and so $I(\RZ_C)\geq `g_\ell$, i.e., larger than all values in the interval.
%	The reverse inequality also holds by \uref{i} and \uref{ii}. Hence,
%	\begin{align*}
%	I(\RZ_C)\utag{iii}=`g_\ell,
%	\end{align*}
%	which implies \eqref{eq:A:`g_l} as desired. 
%
%%	\eqref{eq:A:P_i-1} can be proved by showing that the above construction gives all the optimal solutions to the r.h.s.\ of
%%	\eqref{eq:A:`g_l}.
%	
%	Now, we argue that the above construction gives all the optimal solutions to the r.h.s.\ of
%	\eqref{eq:A:`g_l}, hence establishing \eqref{eq:A:P_i-1}. For any $C\in
%	\mcP_{\ell-1}`/\mcP_{\ell}$, \uref{ii} and \uref{iii} imply that ``$\subseteq$'' holds
%	for~\eqref{eq:A:P_i-1}, because the fact that $C\in \pzC_{`g_{\ell-1}}(\RZ_V)$ (by
%	Proposition~\ref{prop:clusters}) means that it is maximal by the definition~\eqref{eq:clusters}
%	of  clusters. 
%	
%	To argue the reverse inclusion ``$\supseteq$'', consider any $\mcF$ belonging to the r.h.s.\
%	of~\eqref{eq:A:P_i-1}. By \eqref{eq:A:`g_l}, 
%	%\begin{align*}
%	$I(\RZ_{\bigcup \mcF})=`g_\ell>`g_{\ell-1}$
%	%\end{align*}
%	and so $\bigcup \mcF\in \pzC_{`g_{\ell-1}}$ by the definition~\eqref{eq:clusters} of clusters and
%	the maximality of $\bigcup \mcF$. This completes the poof of~\eqref{eq:A:P_i-1}.
%	%To prove \eqref{eq:A:P_i-1}, 
%	
%	\eqref{eq:A:`g_l:fused} and \eqref{eq:A:P_i-1:fused} can be proved using the definition~\eqref{eq:IC*} of $I^*$ and $\pzC^*$ and Proposition~\ref{prop:fund:cluster}.
%\end{Proof}

