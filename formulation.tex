\section{Preliminaries and Problem formulation}
\label{sec:problem}
Consider a random vector $\RZ_V:=(\RZ_i \mid i \in V)$ where $V:=\Set {1,\dots,n}$ is a set of $n>1$
RVs. Let $\Pi(V)$ be the set of partitions of $V$ into non-empty disjoint sets and
$\Pi'(V):=\Pi(V)`/\Set {V}$ be the set of non-trivial partitions.
%%%%%%%%%%%%%
For $\mcP, \mcP' \in \Pi(V)$, we say $\mcP$ is finer than $\mcP'$, denoted $\mcP \preceq \mcP'$, if
\begin{align}
	\label{eq:finer}
	\forall C \in \mcP,  \exists  C' \in \mcP' : C \subseteq C',
\end{align}
and use ``$\prec$" to denote the strict inequality, i.e., when the inclusion above is strict for at least
one $C \in \mcP$.
%%%%%%%%%%%%%%
%The multivariate mutual information (MMI) is characterized in~\cite{chan15mi} as the
%smallest value of $`g\in `R$ such that the \emph{residual~independence relation (RIR)} holds, i.e.,
%for some partition $\mcP\in \Pi'(V)$,
%\begin{subequations}
%\begin{alignat}{2}
%	\label{eq:RIR}
%	h_{`g}(V) &= \sum_{C\in\mcP} h_{`g}[\mcP] &\kern1em& \text {where}\\
%	h(B)&:=H(\RZ_B) && \text{for } B\subseteq V, \label{eq:h}\\
%	h_{`g}(B)&:= h(B)-`g && \text{and} \label{eq:residualH}\\
%	h_{`g}[\mcP] &:= \sum\nolimits_{C\in \mcP} h_{`g}(C) && \text{for $\mcP\in \Pi (V)$}. \label{eq:DT:2} 
%\end{alignat}
%\end{subequations}
%Here, $h$ is the entropy function and $h_{`g}(B)$ is the \emph{residual entropy/randomness} of
%$\RZ_B$ after removing $`g$. The l.h.s.\ of \eqref{eq:RIR} is the total residual randomness of the
%entire set of random variables, and the r.h.s.\ is the sum of the individual residual randomness of
%a partition of the random variables. The equality means that there is independence (no double
%counting) in the individual residual randomness, i.e., $`g$ is the amount of mutual randomness whose
%removal leads to an independence relation.
%%%%%%%%%%%%%%%%%%%%%
%%More explicitly, the MMI, denoted as $I(\RZ_V)$, can equivalently be written as
%This characterization is equivalent to the followoing (explicit) formulation of the MMI, denoted as $I(\RZ_V)$, can be written as
%%%%%%%%%%%%%%
The \emph{multivariate mutual information} (MMI)~\cite{chan15mi} can be defined as
\begin{subequations}
	\label{eq:mmi}
	\begin{align}
		I(\RZ_V)&:=\min_{\mcP\in \Pi'(V)} I_{\mcP}(\RZ_V), \kern1em \text{where} \label{eq:I}\\
		I_{\mcP}(\RZ_V)&:= \frac{1}{|\mcP|-1}`1[\sum\nolimits_{C\in \mcP}H(\RZ_C) - H(\RZ_V) `2].\label{eq:IP}
	\end{align}
\end{subequations}
It follows that the MMI $I(\RZ_V)$ is non-negative, and is equal to zero iff $P_{\RZ_V}=\prod_{C\in \mcP}
P_{\RZ_C}$ for some $\mcP \in \Pi'(V)$.

Subsequently, we will denote the entropy function by $h$, i.e., 
\begin{align}
	h(B)&:=H(\RZ_B) && \text{for } B\subseteq V. \label{eq:h}
\end{align}
%
Essential to this work is the
submodularity of entropy~\cite{fujishige78}
or, equivalently,
the non-negativity of the conditional mutual information~\cite{shannon48}.
More precisely,
the \emph{submodularity} of entropy means that the entropy function $h$ in \eqref{eq:h} satisfies
\begin{align}
	h(B_1)+h(B_2)\geq h(B_1\cup B_2)+h(B_1\cap B_2)  \label{eq:submodular}
\end{align}
for all $B_1,B_2\subseteq V$.
The entropy function is also said to be \emph{normalized} since $h(`0)=0$, and non-decreasing since $h(B')\leq h(B)$ whenever $B' \subseteq B \subseteq V$.

%\begin{example}
%	\label{eg:mmi}
%	Consider the following RVs
%	defined, using the uniformly distributed and independent binary RVs $\RX_a, \RX_b, \RX_c$ and
%	$\RX_d$, as
%	\begin{align}
%		\label{eq:eg-motivate}
%		\begin{array}{lll}
%			\RZ_1:=(\RX_a,\RX_d), &\RZ_2:=(\RX_a,\RX_d), &\RZ_3:=\RX_a, 
%			\\
%			\RZ_4:=\RX_b, &\RZ_5:=\RX_b, &\RZ_6:=\RX_c. 
%		\end{array}
%	\end{align}
%
%	aaa--- continue by listing the MMI for some subsets of V
%\end{example}


Using the MMI, the set of clusters at any threshold $`g\in `R$ is defined in \cite{chan16cluster} as
\begin{align}
	\pzC_{`g}(\RZ_V):=\op{maximal}\{B\subseteq V \mid \abs {B} > 1, I(\RZ_B) > `g\},\label{eq:clusters}
\end{align}
where $\op{maximal}\pzF := \{B \in \pzF \mid \nexists B' \in \pzF, B \subsetneq B' \}$ denotes the
collection of inclusion-wise maximal elements of any collection $\pzF$ of subsets. It was shown in
\cite[Theorem~3]{chan16cluster} that the clusters form a laminar family, i.e.,
for any $`g' \leq `g''$, $C' \in \pzC_{`g'}(\RZ_V)$, and $C'' \in \pzC_{`g''}(\RZ_V)$, we have $C' \cap C'' = `0$ or $C' \supseteq C''$.
%we have
%\begin{align}
%	\label{eq:laminar}
%	C' \cap C'' = `0 \text{ \ or \ } C' \supseteq C'' 
%\end{align}
%for all $`g' \leq `g'', C' \in \pzC_{`g'}(\RZ_V)$, and $C'' \in \pzC_{`g''}(\RZ_V)$.
In particular, clusters at the same threshold must be disjoint. 
Consequently, the clustering solution can be characterized by a sequence of partitions of $V$ as follows.

\begin{Proposition}[\mbox{\cite[Theorems~1 \& 4]{chan16cluster}}]%[\mbox{\cite[Theorems~2.1 and 2.4]{chan16cluster}}]
	\label{prop:clusters}
	\begin{subequations}
	\label{eq:PSP}	
	The clustering solution~\eqref{eq:clusters} satisfies
	\begin{align}
	\kern-7pt	\pzC_{`g}(\RZ_V) &\!=\! \mcP_{\ell} \kern1pt \backslash \kern1pt \{\{i\}\mid i \kern-1pt \in \kern-1pt V\}
	\ \forall \kern1pt %\kern1em \text {for }
		`g \kern-1pt \in \kern-1pt [`g_\ell,`g_{\ell+1}), 0\leq \! \ell \! \leq \! N,
	\end{align}
	where $N$ is a positive integer, %$`g_0:=-`8$ and $`g_{N+1}:=`8$ for convenience;
		\begin{align}
			\label{eq:laminar-gamma-seq}
			`g_0:=-\infty< `g_1 < \cdots < `g_N < `g_{N+1}:=\infty
		\end{align}
		is a sequence of distinct critical values from $`R$ (consisting of the thresholds at which the set of clusters changes), and 
		\begin{align}
			\label{eq:laminar-partitions-seq}
			\kern-8pt
			\mcP_0:=\{V\} \succ \mcP_1  \succ \! \cdots \! \succ \mcP_{N-1} \succ \mcP_{N}\!:=\{\{i\} \mid i\in V\}\kern-.5em
		\end{align}
		\end{subequations}
		is a sequence of increasingly finer partitions of $V$ from $\Pi(V)$.
\end{Proposition}

\begin{remark}
	The proposition above follows from the following property of the MMI
	\begin{align}
		\label{eq:imunion}
		I(\RZ_{B_{1} \cup B_{2}}) \geq \min\{I(\RZ_{B_1}), I(\RZ_{B_2})\}
	\end{align}
	for all $B_{1}, B_{2}\subseteq V:|B_{1}| > 1$, $|B_{2}|>1$, and $B_{1} \cap B_{2} \neq
	\emptyset$.
	In other words, replacing $I(\RZ_{B})$ in \eqref{eq:clusters} with any multivaraite information
	measure that satisfies \eqref{eq:imunion}, the resulting clusters will satisfy the proposition. 
	However, for definiteness, we restrict attention to the MMI \eqref{eq:mmi} in this work.
	In this case, the sequence of partitions in the proposition coincides with the principal sequence
	of partitions (PSP) of the
% 	Dilworth truncation of the
	entropy function, as briefly discussed below. 
\end{remark}

The sequence of partitions in Proposition~\ref{prop:clusters} (together with the critical values) was further shown in
\cite[Corollary~2]{chan16cluster} to be the \emph{principal sequence of partitions} (PSP)
of the entropy function of $\RZ_{V}$, which arises from the Dilworth truncation of the submodular
function $h$~\cite{narayanan90}. (For completeness, we give a brief discussion on the Dilowth
truncation in Appendix~\ref{sec:dt} and refer the interested reader to \cite{chan16cluster,
chan15mi, narayanan90} for more details.)

%%%%%%%%%%%%%%%%%%%%%%%%%%%%%%%%%%%%%%%%%%%%%%%%%%%%%
\begin{example}
	\label{eg:psp}
	As an illustration of Proposition~\ref{prop:clusters} and the PSP, consider the following RVs
	defined, using the uniformly distributed and independent binary RVs $\RX_a, \RX_b, \RX_c$ and
	$\RX_d$, as
	\begin{align}
		\label{eq:eg-motivate}
		\begin{array}{lll}
			\RZ_1:=(\RX_a,\RX_d), &\RZ_2:=(\RX_a,\RX_d), &\RZ_3:=\RX_a, 
			\\
			\RZ_4:=\RX_b, &\RZ_5:=\RX_b, &\RZ_6:=\RX_c. 
		\end{array}
	\end{align}
	The PSP of the entropy function of $\RZ_{\{1,\dots,6\}}$ is shown in \figref{fig:eg-psp}, where the critical values are 
	$	`g_1 = 0, `g_2 = 1,$ and $`g_3 = 2$.%
	\footnote{Here we assumed that the PSP is given since the purpose of the example is to illustrate
		Proposition~\ref{prop:clusters} rather than illustrate the computation of the PSP. Subsequent examples
	will relax this assumption and, in effect, compute the PSP using a  divisive or agglomerative
	approach.}
	From Proposition~\ref{prop:clusters}, the clusters~\eqref{eq:clusters} are given as
	\begin{align}
		\label{eq:eg-clusters}
		\pzC_{`g}(\RZ_{\{1,\ldots,6\}}) = \left\{
			\begin{array}{ll}
				\{\{1,2,3,4,5,6\}\}, 		& `g < 0 \\
				\{\{1,2,3\},\{4,5\}\}, 	& `g \in [0,1) \\
				\{\{1,2\}\},			 	& `g \in [1,2) \\
				\emptyset,				 	& `g \geq 2.
			\end{array}
			\right.\TheoremSymbol
	\end{align}
\end{example}

From Proposition~\ref{prop:clusters}, given the PSP of the entropy function, one can readily obtain the clustering
solution, and vice versa.
%%%%%%%%%%%%%%%%%%%%%%%%%%%%%%%%%%%%%%%%%%%%%%%%%%%%%
There are algorithms for computing the PSP, see e.g.,
\cite[Algorithm~3]{chan16cluster}, \cite[Ch.~13]{narayanan:book}, \cite{nagano10}. However, such algorithms
compute the partitions in the PSP in no particular order. Due to the practical considerations
discussed earlier, it is desirable to have an algorithm that computes the PSP in some order. 
(From $\mcP_1$ to $\mcP_N$, or, preferably, from $\mcP_{N}$ to $\mcP_{1}$.) 



%%%%%%%%%%%%%%%%%%%%%%%%%%%%%%%%%
\begin{figure}
	\begin{center}
		\subcaptionbox{ \label{fig:eg-psp}}{
		\def\thickness{very thick}
		\scalebox{0.8}{
			\input{psp.tex}
		}
		}%
		%\hfill
		%%%%%%%%%%%%%%%%%%%%%%%%
		\subcaptionbox{ \label{fig:eg-div-zv}}{
		\def\thickness{very thick}
		\scalebox{0.8}{
			\input{lattice-Izv.tex}
		}
		}%\hfill
		\subcaptionbox{\label{fig:eg-div-z123}}{
			\def\thickness{very thick}
			\scalebox{0.8}{
			\input{lattice-Izc.tex}
		}
		}%
%		\hfill
%		\subcaptionbox{$\mcP^{*}(\RZ_{\Set{4,5}})$ \label{fig:eg-div-z45}}[3cm]{
%			\input{dtl/dtl-45.tex}
%		}
	\end{center}
	\caption{
%		Optimal partitions of: (a) $I(\RZ_V)$ and (b) $I(\RZ_{\Set{1,2,3}})$. In each case, the
%		fundamental partition is the one at the bottom, where the associated clusters are circled with
%		thick lines.
		(a) The PSP of Example~\ref{eg:psp}, where the non-trivial clusters are circled with thick lines.  
		(b) and (c) Optimal partitions of $I(\RZ_V)$ and  $I(\RZ_{\Set{1,2,3}})$, respectively, of
		Example~\ref{eg:divisive}. In (b) and (c), the
		fundamental partition is the one at the bottom and the associated clusters are circled with
		thick lines.
	}
\label{fig:eg-div}
\end{figure}
%%%%%%%%%%%%%%%%%%%%%%%%%
